\section{Handling structs}

\begin{frame}
  \frametitle{Handling \texttt{struct}s}
  \framesubtitle{What a is a \texttt{struct} in Go}

  \begin{block}{A basic \texttt{struct} in Go}
    \centering
    \inputminted[
      firstline=3,
      lastline=5,
      fontsize=\scriptsize,
      framesep=4mm]
    {go}{go/struct.go}
    \uncover<2->{\inputminted[
      firstline=8,
      lastline=8,
      tabsize=0,
      fontsize=\scriptsize,
      framesep=4mm
    ]{go}{go/struct.go}}
    \uncover<3->{\inputminted[
      firstline=9,
      lastline=9,
      tabsize=0,
      fontsize=\scriptsize,
      framesep=4mm
    ]{go}{go/struct.go}}
    \uncover<4->{\inputminted[
      firstline=10,
      lastline=10,
      tabsize=0,
      fontsize=\scriptsize,
      framesep=4mm
    ]{go}{go/struct.go}}
  \end{block}

  \begin{center}
    \uncover<5->{\textbf{First goal:} Handling basic \texttt{struct}.}\\
    \uncover<5->{\texttt{struct} introduces a large panel of features to manage.}
  \end{center}
\end{frame}

\begin{frame}
  \frametitle{Handling \texttt{struct}s}
  \framesubtitle{How to represent a \texttt{struct} in \gotopins}
  \begin{itemize}
    \item Detect \texttt{struct}s' declaration
    \item Transform \texttt{struct}s' declaration
      \begin{itemize}
        \item Instantiate a \texttt{struct} counter
        \item Convert \texttt{struct} into a number (for further manipulations)
      \end{itemize}
    \item Transform \texttt{struct} variables into an array
  \end{itemize}

  \begin{block}{Transforming a \texttt{struct}}
    \begin{columns}
      \begin{column}{0.49\textwidth}
        \centering
        \inputminted[
          firstline=3,
          fontsize=\scriptsize,
          framesep=4mm]
        {go}{go/structs.go}
      \end{column}
      \begin{column}{0.49\textwidth}
        \centering
        \uncover<2->{\inputminted[
          firstline=3,
          fontsize=\scriptsize,
          framesep=4mm]
        {go}{go/structs-go2pins.go}}
      \end{column}
    \end{columns}
  \end{block}
\end{frame}
